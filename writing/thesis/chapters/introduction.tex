
\chapter{Introduction}
\section{Context and Novelty}

\section{Motivation}

\section{Challenges}
The major challenges of this study were as follows:
\begin{itemize}
    \item Creating a novel augmentation technique that uses attention maps to improve the performance of the model.
    \item Testing the effect of Proxy attention on the explainability of the model.
    \item Comparing a large number of hyperparameters and models with limited computational resources.
    \item Optimizating the usage of XAI techniques to improve the computational efficiency of Proxy Attention.
\end{itemize}

\section{Problem Statement}
The problem statement of this study is creating a novel augmentation technique - \textbf{Proxy Attention}, that uses attention maps to improve the performance of any model by guiding its attention away from the regions that are not important for the classification task. 
In turn this method should also improve the explainability of the model while maintaining the same architecture and hyperparameters.

\section{Research Questions} \label{section:researchq}
The main research questions that summarize the aims of this study are as follows.
\begin{enumerate}
    \item Is it possible to create an augmentation technique that uses Attention maps?
    \item Is it possible to approximate the effects of Attention from ViTs in a CNN without changing the architecture?
    \item Is it possible to make a network converge faster and consequently require less data using the outputs from XAI techniques?
    \item Does using Proxy Attention impact the explainability positively?
\end{enumerate}
\section{Thesis Outline}
This thesis follows the following structure:
\begin{itemize}
    \item \textbf{Chapter 2} provides the necessary background information about the relevant topics and datasets used in this study.
    \item \textbf{Chapter 3} provides a literature review of the relevant topics.
    \item \textbf{Chapter 4} describes the methodology used in this study and the implementation details.
    \item \textbf{Chapter 5} discusses the results and answers the research questions.
    \item \textbf{Chapter 6} concludes the thesis and provides recommendations for future work.
    \item \textbf{Chapter 7} provides additional details and results.
\end{itemize}