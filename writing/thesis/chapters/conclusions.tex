\chapter{Conclusion}
\section{Contributions}
- Novel augmentation technique that uses attention maps
    - Better explainability , better performance
    - Faster convergence, less data
    - Easy to implement, no change in architecture
- Experiments with a large number of hyperparameters and models to test the robustness of the method and find the best configuration
- Open source callback code that can be used to easily add Proxy Attention to any existing code base
- Script to easily parse tensorboard logs to a unified DataFrame for easy analysis
- Scripts to reproduce all the results in this thesis along with training logs

\section{Lessons Learned}
- Combining research from different domains to create a novel method
- Importance of hyperparameter tuning
- Understading memory leaks and debugging them, how to optimize memory usage 
- Importance of writing functional (instead of class oriented) code that can be easily reused and modified 
- Better understanding of Augmentation, XAI techniques
- Better understanding of configuring the training loop

\section{Future Work}
- Schedule both the number of proxy steps, number of images for Proxy step based on validation performance
- Experiment with more XAI methods
- Experiment with smoothing the attention maps before using them for the Proxy step
