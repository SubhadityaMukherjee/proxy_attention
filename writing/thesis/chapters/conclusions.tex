\chapter{Conclusion} \label{ch:conclusion}
\section{Contributions}
% - Novel augmentation technique that uses attention maps
%     - Better explainability , better performance
%     - Faster convergence, less data
%     - Easy to implement, no change in architecture
% - Experiments with a large number of hyperparameters and models to test the robustness of the method and find the best configuration
% - Open source callback code that can be used to easily add Proxy Attention to any existing code base
% - Script to easily parse tensorboard logs to a unified DataFrame for easy analysis
% - Scripts to reproduce all the results in this thesis along with training logs
The contributions of this thesis are as follows:
\begin{itemize}
    \item \textbf{Novel augmentation technique that uses attention maps} We proposed a novel, easy to implement augmentation technique - \textit{Proxy Attention} that uses attention maps generated by XAI methods to emulate Attention in CNNs. We showed that this technique can be used to improve the performance of CNNs without any change in the architecture. We also showed that Proxy Attention improves the explainability of the model with minimal computational overhead.
    \item \textbf{Robustness} We performed a large number of experiments with different hyperparameters and models to test the robustness of the method and find the best configuration. 
    \item \textbf{Open source callback code} We have open sourced the callback code that can be used to easily add Proxy Attention to any existing code base.
    \item \textbf{Tensorboard Log Parser} We have also open sourced a script to easily parse tensorboard logs to a unified DataFrame for easy analysis. This script was used to generate the plots and tables in this thesis.
    \item \textbf{Reproducibility} All the scripts used for this thesis along with the training logs are available for open source. This makes it easy to reproduce all the results in this thesis.
\end{itemize}

\section{Lessons Learned}
% - Combining research from different domains to create a novel method
% - Importance of hyperparameter tuning
% - Understading memory leaks and debugging them, how to optimize memory usage 
% - Importance of writing functional (instead of class oriented) code that can be easily reused and modified 
% - Better understanding of Augmentation, XAI techniques
% - Better understanding of configuring the training loop
The lessons learned from this thesis are as follows:
\begin{itemize}
    \item \textbf{Combining research from different domains to create a novel method} This thesis taught me how to combine research from different domains to create a novel method. In this case, we combined research from the domains of XAI and Augmentation to create a novel augmentation technique.
    \item \textbf{Hyperparameter Tuning} We performed a large number of experiments with different hyperparameters and models to test the robustness of the method and find the best configuration. Doing so taught me the importance of hyperparameter tuning.
    \item \textbf{Memory Leaks} I encountered a lot of memory leaks while working on the code for this thesis, and in the process learned how to debug and fix them. 
    \item \textbf{Function based code} I wrote the code for this thesis in a functional style instead of an object oriented style as a personal experiment (inspired by the success of the Julia programming language). This made it easy to reuse certain parts of the code and modify others. Doing so taught me the importance of writing functional code.
    \item \textbf{Augmentation} I learned a lot about augmentation while working on this thesis. I learned about the different types of augmentations, how to implement them, and how to use them to improve the performance of CNNs.
    \item \textbf{XAI} I also learned a lot about XAI while working on this thesis. 
    \item \textbf{Training Loop} Previous to this thesis, I had only used the training loop provided by higher level libraries. However, for this thesis, I had to implement the training loop from scratch. Doing so taught me a lot about the different components of the training loop and how to configure them for optimal performance and modify them to suit my needs.
\end{itemize}

\section{Future Work}
% - Schedule both the number of Proxy Steps, number of images for Proxy step based on validation performance
% - Experiment with more XAI methods
% - Experiment with smoothing the attention maps before using them for the Proxy step
While the results of this thesis are promising, there is still a lot of room for improvement. The following are some of the possible future directions for this work:
\begin{itemize}
    \item \textbf{Schedules} Currently, the number of Proxy Steps and the number of images used for the Proxy Step are fixed. It would be interesting to schedule both of these based on the validation performance. For example, if the validation performance is not improving, we can increase the number of Proxy Steps and the number of images used for the Proxy Step.
    \item \textbf{More XAI methods} We have only used a tiny subset of XAI methods for this thesis. It would be interesting to experiment with more XAI methods (eg: other methods from the literature survey) and see if they can be used to improve the performance of Proxy Attention.
    \item \textbf{Smoothing Attention Maps} The attention maps generated by the XAI methods are noisy. While no extra smoothing was used in this thesis, it would be useful to experiment with smoothing the attention maps before using them for the Proxy step. An example of a potentially suitable smoothing method is Eigen Smoothing \cite{jacobPyTorchLibraryCAM2021}.
\end{itemize}
